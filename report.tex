\documentclass[a4paper]{report}
\pagestyle{headings}
\usepackage{hyperref}
\usepackage{listings}
\usepackage{graphicx}
\lstset{language=bash}
\lstset{numbers=right}
\lstset{breaklines}
\title{Lab Report for Object Oriented Programming course \newline
 Lab 1: Reversi Game}
\author{Wang, Chen \\ 16307110064 \\ School of Software\\ Fudan University}
\date{\today}
\bibliographystyle{plain}
\begin{document}
\maketitle

\tableofcontents

\chapter{Background Knowledge \& Concepts Required for This Lab}
\section{Reversi Game}
Reversi is a strategy board game for two players, played on an 8×8 uncheckered board. There are sixty-four identical game pieces called disks (often spelled "discs"), which are light on one side and dark on the other. Players take turns placing disks on the board with their assigned color facing up. During a play, any disks of the opponent's color that are in a straight line and bounded by the disk just placed and another disk of the current player's color are turned over to the current player's color. 
\par
The object of the game is to have the majority of disks turned to display your color when the last playable empty square is filled. 
\par
Reversi was most recently marketed by Mattel under the trademark Othello. 

\subsection{Original Version of History}
The game Reversi was invented in 1883 by either of two Englishmen (each claiming the other to be a fraud), Lewis Waterman or John W. Mollett (or perhaps earlier by someone else entirely), and gained considerable popularity in England at the end of the nineteenth century. The game's first reliable mention is in the 21 August 1886 edition of The Saturday Review. Later mention includes an 1895 article in The New York Times: "Reversi is something like Go Bang, and is played with 64 pieces." In 1893, the German games publisher Ravensburger started producing the game as one of its first titles. Two 18th-century continental European books dealing with a game that may or may not be Reversi are mentioned on page fourteen of the Spring 1989 Othello Quarterly, and there has been speculation, so far without documentation, that the game has older origins.

\subsection{Modern Version of History}
The modern version of the game — the most regularly used rule-set, and the one used in international tournaments — is marketed and recognized as Othello. It was patented in Japan in 1971 by Goro Hasegawa (autonym: Satoshi Hasegawa), then a 38-year-old salesman.
\par
There is one difference from the original game: 
\begin{enumerate}
\item The first four pieces go in the center, but in a standard diagonal pattern, rather than being placed by players.
\end{enumerate}
According to Ben Seeley, another difference of Reversi from Othello is that in the first one the game ends as soon as either player cannot make a move, while in the latter the player without a move simply passes.
\par
Hasegawa established the Japan Othello Association on March 1973, and held the first national Othello championship on April 4, 1973 in Japan. The Japanese game company Tsukuda Original launched Othello in late April, 1973 in Japan under Hasegawa’s license, which led to an immediate commercial success.
\par
The name was selected by Hasegawa as a reference to the Shakespearean play Othello, the Moor of Venice, referring to the conflict between the Moor Othello and Iago, and more controversially, to the unfolding drama between Othello, who is black, and Desdemona, who is white. The green color of the board is inspired by the image of the general Othello, valiantly leading his battle in a green field. It can also be likened to a jealousy competition (jealousy being the central theme in Shakespeare's play, which popularized the term "green-eyed monster"), since players engulf the pieces of the opponent, thereby turning them to their possession. 
\par
Othello was first launched in the U.S. in 1975 by Gabriel Industries and it also enjoyed commercial success there. Reportedly, Othello game sales have exceeded $600 million and more than 40 million classic games have been sold in over 100 different countries. 
\par
Hasegawa also wrote How to Othello (Osero No Uchikata) in Japan in 1974, which was later translated into English and published in the U.S. in 1977 as How to Win at Othello.
\par
Kabushiki Kaisha Othello, which is owned by Hasegawa, registered the trademark "OTHELLO" for board games in Japan and Tsukuda Original registered the mark in the rest of the world. All intellectual property regarding Othello outside Japan is now owned by MegaHouse, a Japanese toy company that acquired PalBox, the successor to Tsukuda Original.

\subsection{Rules}

\subsection{Computer opponents and research}




\section{Object-oriented Programming}
\subsection{Features of Object-oriented Programming}



\subsection{History of Object-oriented Programming}


\subsection{OOP languages}


\subsection{Design patterns}

\subsection{Formal semantics}




\chapter{Specifications of This Lab}
\section{Regulations in the game}



\section{Guidelines for computer side piece placing}




\section{Other specifications of the programming}












\chapter{Structure and Object Oriented Ideas Adopted in My Implementation}
\section{Overview}



\section{UML Diagram for the structure of the project}








\chapter{Running Result of My Implementation}






\begin{thebibliography}{A}

\bibitem{1}
Wikipedia contributors. (2018, December 24). Version control. In \emph{Wikipedia, The Free Encyclopedia}. Retrieved 06:12, March 10, 2019, from \url{https://en.wikipedia.org/w/index.php?title=Version_control&oldid=875227317}

\bibitem{2}
Wikipedia contributors. (2019, March 10). Systems development life cycle. In \emph{Wikipedia, The Free Encyclopedia}. Retrieved 06:13, March 10, 2019, from \url{https://en.wikipedia.org/w/index.php?title=Systems_development_life_cycle&oldid=887015682}

\bibitem{3}
Stolen, L. H. (1999). Distributed control system. \emph{international telecommunications energy conference.}

\bibitem{4}
Murayama, T. (1991). Distributed Control System. \emph{international conference on advanced robotics robots in unstructured environments}.

\bibitem{5}
Wikipedia contributors. (2019, March 6). Distributed control system. In \emph{Wikipedia, The Free Encyclopedia}. Retrieved 06:18, March 10, 2019, from \url{https://en.wikipedia.org/w/index.php?title=Distributed_control_system&oldid=886468871}

\end{thebibliography}
\end{document} 